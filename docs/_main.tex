\documentclass[]{book}
\usepackage{lmodern}
\usepackage{amssymb,amsmath}
\usepackage{ifxetex,ifluatex}
\usepackage{fixltx2e} % provides \textsubscript
\ifnum 0\ifxetex 1\fi\ifluatex 1\fi=0 % if pdftex
  \usepackage[T1]{fontenc}
  \usepackage[utf8]{inputenc}
\else % if luatex or xelatex
  \ifxetex
    \usepackage{mathspec}
  \else
    \usepackage{fontspec}
  \fi
  \defaultfontfeatures{Ligatures=TeX,Scale=MatchLowercase}
\fi
% use upquote if available, for straight quotes in verbatim environments
\IfFileExists{upquote.sty}{\usepackage{upquote}}{}
% use microtype if available
\IfFileExists{microtype.sty}{%
\usepackage{microtype}
\UseMicrotypeSet[protrusion]{basicmath} % disable protrusion for tt fonts
}{}
\usepackage[margin=1in]{geometry}
\usepackage{hyperref}
\hypersetup{unicode=true,
            pdftitle={fulltext manual},
            pdfborder={0 0 0},
            breaklinks=true}
\urlstyle{same}  % don't use monospace font for urls
\usepackage{natbib}
\bibliographystyle{apalike}
\usepackage{color}
\usepackage{fancyvrb}
\newcommand{\VerbBar}{|}
\newcommand{\VERB}{\Verb[commandchars=\\\{\}]}
\DefineVerbatimEnvironment{Highlighting}{Verbatim}{commandchars=\\\{\}}
% Add ',fontsize=\small' for more characters per line
\usepackage{framed}
\definecolor{shadecolor}{RGB}{248,248,248}
\newenvironment{Shaded}{\begin{snugshade}}{\end{snugshade}}
\newcommand{\AlertTok}[1]{\textcolor[rgb]{0.94,0.16,0.16}{#1}}
\newcommand{\AnnotationTok}[1]{\textcolor[rgb]{0.56,0.35,0.01}{\textbf{\textit{#1}}}}
\newcommand{\AttributeTok}[1]{\textcolor[rgb]{0.77,0.63,0.00}{#1}}
\newcommand{\BaseNTok}[1]{\textcolor[rgb]{0.00,0.00,0.81}{#1}}
\newcommand{\BuiltInTok}[1]{#1}
\newcommand{\CharTok}[1]{\textcolor[rgb]{0.31,0.60,0.02}{#1}}
\newcommand{\CommentTok}[1]{\textcolor[rgb]{0.56,0.35,0.01}{\textit{#1}}}
\newcommand{\CommentVarTok}[1]{\textcolor[rgb]{0.56,0.35,0.01}{\textbf{\textit{#1}}}}
\newcommand{\ConstantTok}[1]{\textcolor[rgb]{0.00,0.00,0.00}{#1}}
\newcommand{\ControlFlowTok}[1]{\textcolor[rgb]{0.13,0.29,0.53}{\textbf{#1}}}
\newcommand{\DataTypeTok}[1]{\textcolor[rgb]{0.13,0.29,0.53}{#1}}
\newcommand{\DecValTok}[1]{\textcolor[rgb]{0.00,0.00,0.81}{#1}}
\newcommand{\DocumentationTok}[1]{\textcolor[rgb]{0.56,0.35,0.01}{\textbf{\textit{#1}}}}
\newcommand{\ErrorTok}[1]{\textcolor[rgb]{0.64,0.00,0.00}{\textbf{#1}}}
\newcommand{\ExtensionTok}[1]{#1}
\newcommand{\FloatTok}[1]{\textcolor[rgb]{0.00,0.00,0.81}{#1}}
\newcommand{\FunctionTok}[1]{\textcolor[rgb]{0.00,0.00,0.00}{#1}}
\newcommand{\ImportTok}[1]{#1}
\newcommand{\InformationTok}[1]{\textcolor[rgb]{0.56,0.35,0.01}{\textbf{\textit{#1}}}}
\newcommand{\KeywordTok}[1]{\textcolor[rgb]{0.13,0.29,0.53}{\textbf{#1}}}
\newcommand{\NormalTok}[1]{#1}
\newcommand{\OperatorTok}[1]{\textcolor[rgb]{0.81,0.36,0.00}{\textbf{#1}}}
\newcommand{\OtherTok}[1]{\textcolor[rgb]{0.56,0.35,0.01}{#1}}
\newcommand{\PreprocessorTok}[1]{\textcolor[rgb]{0.56,0.35,0.01}{\textit{#1}}}
\newcommand{\RegionMarkerTok}[1]{#1}
\newcommand{\SpecialCharTok}[1]{\textcolor[rgb]{0.00,0.00,0.00}{#1}}
\newcommand{\SpecialStringTok}[1]{\textcolor[rgb]{0.31,0.60,0.02}{#1}}
\newcommand{\StringTok}[1]{\textcolor[rgb]{0.31,0.60,0.02}{#1}}
\newcommand{\VariableTok}[1]{\textcolor[rgb]{0.00,0.00,0.00}{#1}}
\newcommand{\VerbatimStringTok}[1]{\textcolor[rgb]{0.31,0.60,0.02}{#1}}
\newcommand{\WarningTok}[1]{\textcolor[rgb]{0.56,0.35,0.01}{\textbf{\textit{#1}}}}
\usepackage{longtable,booktabs}
\usepackage{graphicx,grffile}
\makeatletter
\def\maxwidth{\ifdim\Gin@nat@width>\linewidth\linewidth\else\Gin@nat@width\fi}
\def\maxheight{\ifdim\Gin@nat@height>\textheight\textheight\else\Gin@nat@height\fi}
\makeatother
% Scale images if necessary, so that they will not overflow the page
% margins by default, and it is still possible to overwrite the defaults
% using explicit options in \includegraphics[width, height, ...]{}
\setkeys{Gin}{width=\maxwidth,height=\maxheight,keepaspectratio}
\IfFileExists{parskip.sty}{%
\usepackage{parskip}
}{% else
\setlength{\parindent}{0pt}
\setlength{\parskip}{6pt plus 2pt minus 1pt}
}
\setlength{\emergencystretch}{3em}  % prevent overfull lines
\providecommand{\tightlist}{%
  \setlength{\itemsep}{0pt}\setlength{\parskip}{0pt}}
\setcounter{secnumdepth}{5}
% Redefines (sub)paragraphs to behave more like sections
\ifx\paragraph\undefined\else
\let\oldparagraph\paragraph
\renewcommand{\paragraph}[1]{\oldparagraph{#1}\mbox{}}
\fi
\ifx\subparagraph\undefined\else
\let\oldsubparagraph\subparagraph
\renewcommand{\subparagraph}[1]{\oldsubparagraph{#1}\mbox{}}
\fi

%%% Use protect on footnotes to avoid problems with footnotes in titles
\let\rmarkdownfootnote\footnote%
\def\footnote{\protect\rmarkdownfootnote}

%%% Change title format to be more compact
\usepackage{titling}

% Create subtitle command for use in maketitle
\newcommand{\subtitle}[1]{
  \posttitle{
    \begin{center}\large#1\end{center}
    }
}

\setlength{\droptitle}{-2em}
  \title{fulltext manual}
  \pretitle{\vspace{\droptitle}\centering\huge}
  \posttitle{\par}
  \author{}
  \preauthor{}\postauthor{}
  \predate{\centering\large\emph}
  \postdate{\par}
  \date{2017-12-19 - fulltext v0.1.9.9621}

\usepackage{booktabs}
\usepackage{amsthm}
\makeatletter
\def\thm@space@setup{%
  \thm@preskip=8pt plus 2pt minus 4pt
  \thm@postskip=\thm@preskip
}
\makeatother

\usepackage{amsthm}
\newtheorem{theorem}{Theorem}[chapter]
\newtheorem{lemma}{Lemma}[chapter]
\theoremstyle{definition}
\newtheorem{definition}{Definition}[chapter]
\newtheorem{corollary}{Corollary}[chapter]
\newtheorem{proposition}{Proposition}[chapter]
\theoremstyle{definition}
\newtheorem{example}{Example}[chapter]
\theoremstyle{definition}
\newtheorem{exercise}{Exercise}[chapter]
\theoremstyle{remark}
\newtheorem*{remark}{Remark}
\newtheorem*{solution}{Solution}
\begin{document}
\maketitle

{
\setcounter{tocdepth}{1}
\tableofcontents
}
\hypertarget{fulltext-manual}{%
\chapter{fulltext manual}\label{fulltext-manual}}

\begin{quote}
An R package to search across and get full text for open access journals
\end{quote}

The \texttt{fulltext} package makes it easy to do text-mining by
supporting the following steps:

\begin{itemize}
\tightlist
\item
  Search for articles
\item
  Fetch articles
\item
  Get links for full text articles (xml, pdf)
\item
  Extract text from articles / convert formats
\item
  Collect bits of articles that you actually need
\item
  Download supplementary materials from papers
\end{itemize}

\hypertarget{info}{%
\section{Info}\label{info}}

\begin{itemize}
\tightlist
\item
  Code: \url{https://github.com/ropensci/fulltext/}
\item
  Issues: \url{https://github.com/ropensci/fulltext/issues}
\item
  CRAN: \url{https://cran.rstudio.com/web/packages/fulltext/}
\end{itemize}

\hypertarget{citing-fulltext}{%
\section{Citing fulltext}\label{citing-fulltext}}

\begin{quote}
Scott Chamberlain \& Will Pearse (2017). fulltext: Full Text of
`Scholarly' Articles Across Many Data Sources. R package version
0.1.9.9621. \url{https://github.com/ropensci/fulltext}
\end{quote}

\hypertarget{installation}{%
\section{Installation}\label{installation}}

Stable version from CRAN

\begin{Shaded}
\begin{Highlighting}[]
\KeywordTok{install.packages}\NormalTok{(}\StringTok{"fulltext"}\NormalTok{)}
\end{Highlighting}
\end{Shaded}

Development version from GitHub

\begin{Shaded}
\begin{Highlighting}[]
\NormalTok{devtools}\OperatorTok{::}\KeywordTok{install_github}\NormalTok{(}\StringTok{"ropensci/fulltext"}\NormalTok{)}
\end{Highlighting}
\end{Shaded}

Load library

\begin{Shaded}
\begin{Highlighting}[]
\KeywordTok{library}\NormalTok{(}\StringTok{'fulltext'}\NormalTok{)}
\end{Highlighting}
\end{Shaded}

\hypertarget{intro}{%
\chapter{Introduction}\label{intro}}

\hypertarget{user-interface}{%
\section{User interface}\label{user-interface}}

Functions in \texttt{fulltext} are setup to make the package as easy to
use as possible. The functions are organized around use cases:

\begin{itemize}
\tightlist
\item
  Search for articles
\item
  Get full text links
\item
  Get articles
\item
  Get abstracts
\item
  Pull out article sections of interest
\end{itemize}

Because there are so many data sources for scholarly texts, it makes a
lot of sense to simplify the details of each data source, and present a
single user interface to all of them.

\hypertarget{data-sources}{%
\chapter{Data sources}\label{data-sources}}

Data sources in \texttt{fulltext} include:

\begin{itemize}
\tightlist
\item
  \href{http://www.crossref.org/}{Crossref} - via the \texttt{rcrossref}
  package
\item
  \href{https://www.plos.org/}{Public Library of Science (PLOS)} - via
  the \texttt{rplos} package
\item
  \href{http://www.biomedcentral.com/}{Biomed Central}
\item
  \href{https://arxiv.org}{arXiv} - via the \texttt{aRxiv} package
\item
  \href{http://biorxiv.org/}{bioRxiv} - via the \texttt{biorxivr}
  package
\item
  \href{http://www.ncbi.nlm.nih.gov/}{PMC/Pubmed via Entrez} - via the
  \texttt{rentrez} package
\item
  Many more are supported via the above sources (e.g., \emph{Royal
  Society Open Science} is available via Pubmed)
\item
  We \textbf{will} add more, as publishers open up, and as we have
  time\ldots{}See the
  \href{https://github.com/ropensci/fulltext/issues/4\#issuecomment-52376743}{master
  list here}
\end{itemize}

\hypertarget{authentication}{%
\chapter{Authentication}\label{authentication}}

Some data sources require authentication. Here's a breakdown of how to
do authentication by data source:

\begin{itemize}
\tightlist
\item
  \textbf{BMC}: BMC is integrated into Springer Publishers now, and that
  API requires an API key. Get your key by signing up at
  \url{https://dev.springer.com/}, then you'll get a key. Pass the key
  to a named parameter \texttt{key} to \texttt{bmcopts}. Or, save your
  key in your \texttt{.Renviron} file as \texttt{SPRINGER\_KEY}, and
  we'll read it in for you, and you don't have to pass in anything.
\item
  \textbf{Scopus}: Scopus requires an API key to search their service.
  Go to \url{https://dev.elsevier.com/index.html}, register for an
  account, then when you're in your account, create an API key. Pass in
  as variable \texttt{key} to \texttt{scopusopts}, or store your key
  under the name \texttt{ELSEVIER\_SCOPUS\_KEY} as an environment
  variable in \texttt{.Renviron}, and we'll read it in for you. See
  \texttt{?Startup} in R for help.
\item
  \textbf{Microsoft}: Get a key by creating an Azure account at
  \url{https://www.microsoft.com/cognitive-services/en-us/subscriptions},
  then requesting a key for \textbf{Academic Knowledge API} within
  \textbf{Cognitive Services}. Store it as an environment variable in
  your \texttt{.Renviron} file - see {[}Startup{]} for help. Pass your
  API key into \texttt{maopts} as a named element in a list like
  \texttt{list(key\ =\ Sys.getenv(\textquotesingle{}MICROSOFT\_ACADEMIC\_KEY\textquotesingle{}))}
\item
  \textbf{Crossref}: Crossref encourages requests with contact
  information (an email address) and will forward you to a dedicated API
  cluster for improved performance when you share your email address
  with them.
  \url{https://github.com/CrossRef/rest-api-doc\#good-manners--more-reliable-service}
  To pass your email address to Crossref via this client, store it as an
  environment variable in \texttt{.Renviron} like
  \texttt{crossref\_email\ =\ name@example.com}
\end{itemize}

None needed for \textbf{PLOS}, \textbf{eLife}, \textbf{arxiv},
\textbf{biorxiv}, \textbf{Euro PMC}, or \textbf{Entrez} (though soon you
will get better rate limtits with auth for Entrez)

\hypertarget{search}{%
\chapter{Search}\label{search}}

Search is what you'll likely start with for a number of reasons. First,
search functionality in \texttt{fulltext} means that you can start from
searching on words like `ecology' or `cellular' - and the output of that
search can be fed downstream to the next major task: fetching articles.

\hypertarget{usage}{%
\section{Usage}\label{usage}}

\begin{Shaded}
\begin{Highlighting}[]
\KeywordTok{library}\NormalTok{(fulltext)}
\end{Highlighting}
\end{Shaded}

List backends available

\begin{Shaded}
\begin{Highlighting}[]
\KeywordTok{ft_search_ls}\NormalTok{()}
\end{Highlighting}
\end{Shaded}

\begin{verbatim}
#> [1] "arxiv"      "biorxivr"   "bmc"        "crossref"   "entrez"    
#> [6] "europe_pmc" "ma"         "plos"       "scopus"
\end{verbatim}

Search - by default searches against PLOS (Public Library of Science)

\begin{Shaded}
\begin{Highlighting}[]
\NormalTok{res <-}\StringTok{ }\KeywordTok{ft_search}\NormalTok{(}\DataTypeTok{query =} \StringTok{"ecology"}\NormalTok{)}
\end{Highlighting}
\end{Shaded}

The output of \texttt{ft\_search} is a \texttt{ft} S3 object, with a
summary of the results:

\begin{Shaded}
\begin{Highlighting}[]
\NormalTok{res}
\end{Highlighting}
\end{Shaded}

\begin{verbatim}
#> Query:
#>   [ecology] 
#> Found:
#>   [PLoS: 41094; BMC: 0; Crossref: 0; Entrez: 0; arxiv: 0; biorxiv: 0; Europe PMC: 0; Scopus: 0; Microsoft: 0] 
#> Returned:
#>   [PLoS: 10; BMC: 0; Crossref: 0; Entrez: 0; arxiv: 0; biorxiv: 0; Europe PMC: 0; Scopus: 0; Microsoft: 0]
\end{verbatim}

and has slots for each data source:

\begin{Shaded}
\begin{Highlighting}[]
\KeywordTok{names}\NormalTok{(res)}
\end{Highlighting}
\end{Shaded}

\begin{verbatim}
#> [1] "plos"     "bmc"      "crossref" "entrez"   "arxiv"    "biorxiv" 
#> [7] "europmc"  "scopus"   "ma"
\end{verbatim}

Get data for a single source

\begin{Shaded}
\begin{Highlighting}[]
\NormalTok{res}\OperatorTok{$}\NormalTok{plos}
\end{Highlighting}
\end{Shaded}

\begin{verbatim}
#> Query: [ecology] 
#> Records found, returned: [41094, 10] 
#> License: [CC-BY] 
#>                              id
#> 1  10.1371/journal.pone.0001248
#> 2  10.1371/journal.pone.0059813
#> 3  10.1371/journal.pone.0155019
#> 4  10.1371/journal.pone.0080763
#> 5  10.1371/journal.pone.0150648
#> 6  10.1371/journal.pcbi.1003594
#> 7  10.1371/journal.pone.0102437
#> 8  10.1371/journal.pone.0175014
#> 9  10.1371/journal.pone.0166559
#> 10 10.1371/journal.pone.0054689
\end{verbatim}

\hypertarget{links}{%
\chapter{Links}\label{links}}

links

\hypertarget{fetch}{%
\chapter{Fetch}\label{fetch}}

fetch

\hypertarget{chunks}{%
\chapter{Chunks}\label{chunks}}

chunks

\hypertarget{supplementary}{%
\chapter{Supplementary}\label{supplementary}}

supplementary

\hypertarget{use-cases}{%
\chapter{Use cases}\label{use-cases}}

use cases

\hypertarget{literature}{%
\chapter{Literature}\label{literature}}

Here is a review of existing methods.

\bibliography{book.bib,packages.bib}


\end{document}
